% This is a file that contains the tracking of the activities related to the project during the second week of March 2025.

\documentclass[12pt]{article}

\usepackage[a4paper, margin=1in]{geometry}
\usepackage{graphicx}
\usepackage[backend=biber,style=numeric]{biblatex}
\usepackage[most]{tcolorbox}

\addbibresource{references.bib}

\begin{document}

\author{Joan Ronquillo}

\title{Project Progress - Week 2 of March 2025}
\maketitle

This is a file that contains the tracking of the activities
related to the Hydrodynamic Interactions project during the
second week of March 2025. The activities are divided into
groups and a summary of the progress is provided.

\section{Initial Status of the Project}
The project has made significant progress during the first week of March 2025. The key achievements include:
\begin{itemize}
    \item Implementation of the \path{.gitignore}, \path{setup.py}, and \path{__init__.py} files for modular development.
    \item Creation, check and improvement of the \path{test_RPY_distance.py} script to obtain the RPY mobility for two particles as a function of the distance between them.
    \item Meeting with Rafa to discuss the creation of specific particle configurations and geometries, particularly the configuration of a sphere near a horizontal wall.
\end{itemize}
So far, the project is developing in some areas:
\begin{itemize}
    \item The development of the Python module with the implementation functions for mobility tensor calculations.
    \item The development of theory to express vectors in VSH basis.
    \item The initial functions to establish specific particle arrangements and geometries.
\end{itemize}

\section{Potential Tasks for the Week}
The following tasks are proposed for the week of March 10th to March 15th, 2025:
\begin{itemize}
    \item Modification of the structure of the \path{get_mobility_tensor} function, establishing default arguments and eliminating external initialization.
    \item Reading of Raúl's documents on software development.
    \item Exploration/creation of Python functions that allow specific particle arrays and geometries to be established.
    \item Exploration/creation of Python functions that handle vector spherical harmonics (VSH) and their properties. Study of the convenience of a class.
    \item Creation of Python functions that allow the mobility tensor to be obtained in the basis of VSH.
    \item Brenner's paper \cite{BRENNER1961242} reading.
\end{itemize}

\section{Week Progress}

\subsection{Monday, March 10th, 2025}
The project status and potential tasks for the week have been established.
The definition/interface (arguments) of the
\texttt{get\_mobility\_tensor} function has been updated
and corrected by making a copy with the name
\texttt{get\_mobility\_tensor\_RPY}. The external initialization
of the variables has been removed and default values have been
set for the arguments. The \texttt{test\_RPY\_distance.py}
script tests this function.

\subsection{Wednesday, March 12th, 2025}
% Se ha establecido una estrictura preliminar de modulos para la clase de partículas (futuriblemebnte con metodos para crear arreglos. Se han econtrado varios problemas con la configuración de un paquete general para src, a partir del stup.py, y la actualización de las depenedencias en el entorno virtual creado, habiendo modificado el environment.yml)
A preliminary structure of modules has been established for the
particle class (possibly with methods to create arrays). Several
problems have been found with the configuration of a general package
for \texttt{src} from \texttt{setup.py}, and the update of
dependencies in the created virtual environment, having modified
the \texttt{environment.yml}.

\section{Next Steps}
The next steps for the project are:
\begin{itemize}
    \item Reading of Raúl's documents on software development.
    \item Exploration/creation of Python functions that allow specific particle arrays and geometries to be established.
    \item Exploration/creation of Python functions that handle vector spherical harmonics (VSH) and their properties. Study of the convenience of a class.
    \item Creation of Python functions that allow the mobility tensor to be obtained in the basis of VSH.
    \item Reading of Brenner's paper \cite{BRENNER1961242}.
\end{itemize}

\printbibliography

\end{document}
