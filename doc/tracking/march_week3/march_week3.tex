% This is a file that contains the tracking of the activities related to the project during the third week of March 2025.

\documentclass[12pt]{article}

\usepackage[a4paper, margin=1in]{geometry}
\usepackage{graphicx}
\usepackage[backend=biber,style=numeric]{biblatex}
\usepackage[most]{tcolorbox}

\addbibresource{references.bib}

\begin{document}

\author{Joan Ronquillo}

\title{Project Progress - Week 3 of March 2025}
\maketitle

This is a file that contains the tracking of the activities
related to the Hydrodynamic Interactions project during the
third week of March 2025. The activities are divided into
groups and a summary of the progress is provided.

\section{Initial Status of the Project}
The project has made significant progress during the second week of March 2025. The key achievements include:
\begin{itemize}
    \item Interface of the \path{get_mobility_tensor} function has been updated and corrected.
    \item Preliminary structure of modules for the particle class has been established.
\end{itemize}
So far, the project is developing in some areas:
\begin{itemize}
    \item The development of the Python module with the implementation functions for mobility tensor calculations.
    \item The development of theory to express vectors in VSH basis.
    \item The initial functions to establish specific particle arrangements and geometries.
\end{itemize}

\section{Potential Tasks for the Week}
The following tasks are proposed for the week of March 10th to March 15th, 2025:
\begin{itemize}
    \item Check of the \path{setup.py} installing issues.
    \item Reading of Raúl's documents on software development.
    \item Exploration/creation of Python functions that allow specific particle arrays and geometries to be established.
    \item Exploration/creation of Python functions that handle vector spherical harmonics (VSH) and their properties. Study of the convenience of a class.
    \item Creation of Python functions that allow the mobility tensor to be obtained in the basis of VSH.
    \item Brenner's paper \cite{BRENNER1961242} reading.
\end{itemize}

\section{Week Progress}

\subsection{Wednesday, March 19th, 2025}
The week tasks and current progress have been reviewed. 


\section{Next Steps}
The next steps for the project are:
\begin{itemize}
    \item Check of the \path{setup.py} installing issues.
    \item Reading of Raúl's documents on software development.
    \item Exploration/creation of Python functions that allow specific particle arrays and geometries to be established.
    \item Exploration/creation of Python functions that handle vector spherical harmonics (VSH) and their properties. Study of the convenience of a class.
    \item Creation of Python functions that allow the mobility tensor to be obtained in the basis of VSH.
    \item Reading of Brenner's paper \cite{BRENNER1961242}.
\end{itemize}

\printbibliography

\end{document}
