% This is a file that contains the tracking of the activities related to the project during the first week of March 2025.
% The activities are divided into groups and a summary of the progress is provided.

\documentclass[12pt]{article} % 12pt font size
\usepackage[a4paper, margin=1in]{geometry} % Adjust margins

\usepackage{graphicx}
\usepackage{dirtree}
\usepackage{hyperref}

\begin{document}
\author{Joan Ronquillo}

\title{Project Progress - Week 1 of March 2025}
\maketitle

This is a file that contains the tracking of the activities related to the Hydrodynamic
Interactions project during the first week of March 2025. The activities are divided into
groups and a summary of the progress is provided.

\section{Initial Status of the Project}
The project has made significant progress up until the first week of March 2025. The key achievements include:

\begin{itemize}
    \item Meetings with Raúl to discuss the hydrodynamic interactions.
    \item Documentation in Notion of the basic theory behind hydrodynamic interactions.
    \item Preliminary exploration of the "spreadinterp" repository.
    \item Creation of a general function to obtain the mobility tensor given a solver and a set of particle positions.
    \item Initial tests for obtaining the "Self-Mobility Tensor".
\end{itemize}

\section{Potential Tasks for the Week}
The following tasks have been identified as potential areas of focus for the first week of March 2025:
\begin{itemize}
    \item Development of the Python module with the implementation functions.
    \item Tests for obtaining the RPY tensor. Discussion of the representation and its properties.
    \item Initial functions to establish specific particle arrangements and geometries.
\end{itemize}

\section{Week Tracking}
\subsection{Monday - March 3, 2025}
The repository has been reorganized, and the functionalities of \path{.gitignore}, \path{setup.py},
and \path{__init__.py} have been discussed to manage the import of functions and modules. 
The use of pytest and the inclusion of asserts in the test functions are emphasized. 
The correct functioning of the self-mobility tensor test is verified.

\subsection{Tuesday - March 4, 2025}
I have been learning how to compile LaTeX projects located inside the repository.
The pdf tab viewer in VSCode has been installed and configured to facilitate the 
visualization of the documents. This document serves as an example of the compilation 
process.

\subsection{Wednesday - March 5, 2025}
The tasks for the week have been specified and the progress in documentation, code, and testing has been tracked.
The script \path{test/test_RPY_distance.py} has been created to obtain 
the RPY mobility for two particles as a function of the distance 
between them. The script checks the symmetry of the tensor, the 
reproduction of the self-mobility elements on the diagonal, the 
symmetry of the elements of the off-diagonal blocks of cross mobility,
the nullity of the elements corresponding to crossed coordinates 
(the particles are on the x-axis), and the equivalence between the 
two yy, zz terms (perpendicular to the axis that joins the particles) 
of the cross mobility diagonal. The script also generates a graph
with the dependence of the non-zero elements of the cross mobility
with the distance between the particles, from 0.1 to 10 times the
hydrodynamic radius of the particles. The ordering and storage of
this type of graphs in the repository must still be studied.

\subsection{Thursday - March 6, 2025}
Added checks to the test file \path{test_RPY_distance.py} that
include the equivalence between the analytical result and the matrix
obtained in the numerical calculation. The code has been refactored
to separate functions, with one function for generating the graph and
another for the array of analytical tensors. Additionally, the
\path{output/} directory has been created to store the graphs generated
by the test scripts.

\section{Current Next Steps}
The next steps for the project include:
\begin{itemize} 
    \item Analysys of RPY mobility for two particles as a function of the distance between them.
    \item Development of the Python module with the implementation functions.
\end{itemize}
\end{document}
